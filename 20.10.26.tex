\section{Примеры решения линейных систем с плотными матрицами}

\[
    A x = b, \quad A \in \CC^{n \times n}, \textrm{det}(A) \ne 0,
    \quad b \in \CC^n
\]

\[
    A = Q R, \quad Q R x = b; R x = Q^* b, \quad \frac{4}{3} n^3 + O(n^2)
\]

\[
    A = U \Sigma V^*, \quad x = V \Sigma^{-1} U^* b
\]

\subsection{LU-разложение}

\begin{definition}
    $A = LU$ --- LU-разложение матрицы $A$, где $L$ --- нижнетреугольная с
    1 на диагонали, а $U$ --- верхнетреугольная.
\end{definition}

\[
    L U x = b \quad\Leftrightarrow\quad \begin{cases}
        L y = b; \\ U x = y.
    \end{cases}
\]

Сложность --- $O(n^2)$ на решение системы, $O(n^3)$ на разложение (но
константа меньше, чем в QR).

\begin{theorem}
    Пусть $\textrm{det}(A) \ne 0$. Тогда $A$ имеет LU-разложение, что
    эквивалентно тому, что все ведущие подматрицы невырождены.
\end{theorem}

\begin{proof} .
    \begin{itemize}
        \item $\Rightarrow$ $A = LU$
            \[
                0 \ne \textrm{det}(A) = \textrm{det}(L U) = \textrm{det}(L)
                \textrm{det}(U) = u_{11} \cdot \ldots \cdot u_{nn}
            \]

            Из чего следует, что $\forall k: u_{k k} \ne 0$.

            \[
                A = \begin{pmatrix} L_k & 0 \\ * & * \end{pmatrix}
                \begin{pmatrix} U_k & * \\ 0 & * \end{pmatrix}
                = \begin{pmatrix} L_k U_k & * \\ * & * \end{pmatrix}
            \]

            $\textrm{det}(L_k U_k) = u_{11} \cdot \ldots \cdot u_{kk} \ne 0$.

        \item $\Leftarrow$ --- по индукции.
            \[
                A = \begin{pmatrix} a & c^\top \\ b & D \end{pmatrix}
            \]

            \[
                \begin{pmatrix} 1 & 0 \\ -\frac{b}{a} & I \end{pmatrix}
                \begin{pmatrix} a & c^\top \\ b & D \end{pmatrix}
                = \begin{pmatrix} a & c^\top \\ 0 & D - \frac{b}{a} c^\top
                \end{pmatrix}
            \]

            У $A_1$ все ведущие нормы невырождены (ДЗ).

            По индукции $A_1 = L_1 U_1$.

            \[
                \begin{pmatrix} 1 & 0 \\ \frac{1}{a} b & L_1 \end{pmatrix}
                \begin{pmatrix} a & c^\top \\ 0 & U_1 \end{pmatrix}
                = \begin{pmatrix} a & c^\top \\ \dots \end{pmatrix}
                = \begin{pmatrix} a & c^\top \\ b & D \end{pmatrix}
            \]
    \end{itemize}
\end{proof}

\begin{point}
    LU-разложение определяется единственным образом.
\end{point}

\begin{proof}
    \[
        A = L_1 U_1 = L_2 U_2 \quad\Leftrightarrow\quad
        L_2^{-1} L_1 = U_2 U_1^{-1} = I \quad\Leftrightarrow\quad
        L_1 = L_2,\; U_1 = U_2.
    \]
\end{proof}

\begin{point}
    (LDL-разложение) Пусть у $A$ все ведущие подматрицы невырождены, $A = A^*$,
    $\exists L$ --- нижняя унитреугольная и $D$ --- диагональная:

    \[
        A = L D L^*.
    \]
\end{point}

\begin{proof}
    \[
        A = L U = L D D^{-1} U = A^* = U^* D^{-*} D^* L^*
    \]

    Из единственности следует $L = U^* D^{-*}$, $U^* = L D^*$, $U = D L^*$.
\end{proof}

\subsubsection{Связь LU-разложения и метода исключения Гаусса}

LU:

\[
    \begin{pmatrix} 1 & 0 \\ -\frac{1}{a}b & I \end{pmatrix}
    \begin{pmatrix} a & c^\top \\ b & D \end{pmatrix}
    = \begin{pmatrix} a & c^\top \\ 0 & D - \frac{1}{a} b c^\top \end{pmatrix}
\]

Гаусс:

\[
    \begin{pmatrix}
        1 & 0 & 0 \dots 0 \\
        0 & 1 & 0 \dots 0 \\
        0 & v & I
    \end{pmatrix}
    \begin{pmatrix}
        * & * & \dots & * \\
        0 & * & . & \vdots \\
        0 & * & . & \vdots \\
        \vdots & \vdots & . & \vdots \\
        0 & * & \dots & *
    \end{pmatrix}
    = \begin{pmatrix}
        * & * & \dots & * \\
        0 & * & . & \vdots \\
        0 & 0 & . & \vdots \\
        \vdots & \vdots & . & \vdots \\
        0 & 0 & *\dots & *
    \end{pmatrix}
\]

Сложность LU-разложения --- $\frac{2}{3} n^3 + O(n^2)$.

На практике лучше использовать блочное разложение:

\[
    \begin{pmatrix} A & B \\ C & D \end{pmatrix}
    = \begin{pmatrix} I & 0 \\ C A^{-1} & I \end{pmatrix}
    \begin{pmatrix} A & B \\ 0 & S \end{pmatrix},
    \quad S = D - B A^{-1} C
\]

($S$ --- дополнение по Шуру блока $D$)

\subsection{Выбор ведущего элемента}

\begin{theorem}
    Пусть $\exists$ LU-разложение $A$ и не возникает ???. Тогда

    \[
        |A - \tilde{L} \tilde{U}| \le 3 n \eps_{\text{machine}} (|A| + |L| |U|)
        + O(\eps_{\text{machine}}^2)
    \]
\end{theorem}

\begin{example}
    \[
        \begin{pmatrix} \eps & 1 \\ 1 & 1 \end{pmatrix}
        = \begin{pmatrix} 1 & 0 \\ \frac{1}{\eps} & 1 \end{pmatrix}
        \begin{pmatrix} \eps & 1 \\ 0 & 1 - \frac{1}{\eps} \end{pmatrix}
    \]
\end{example}

\begin{itemize}
    \item Выбор ведущего элемента (partial pivoting)
        \[
            P A = L U
        \]

        (TODO: матрица $P A$)

        $P$ (перестановка) выбирается так, чтобы $a_k$ был максимальным по
        модулю в 1-м столбце $A_k$. Тогда $\norm{L}_c \le 1$. Но элементы в $U$
        ещё могут расти:

        \[
            A = \begin{pmatrix}
                1 & 0 & 0 \dots 0 & 1 \\
                -1 & \ddots & 0 \dots 0 & 1 \\
                \vdots & \ddots & \ddots & \vdots \\
                -1 & \dots & -1 & 1
            \end{pmatrix},
            \quad \frac{\norm{U}_c}{\norm{A}_c} = 2^{n - 1}
        \]

    \item Полный выбор (full pivoting)
        \[
            P A Q = L U
        \]

        Выбираем перестановку такую, чтобы $a_k$ был максимальным по модулю во
        всей $A_k$.
\end{itemize}

\subsection{Разложение Халецкого}

\begin{definition}
    $A = L L^*$ --- разложение Халецкого, где $L$ --- нижнетреугольная матрица.
\end{definition}

\begin{theorem}
    $A$ имеет разложение Халецкого $\Leftrightarrow$ $A = A^* > 0$.
\end{theorem}

\begin{proof} .
    \begin{itemize}
        \item $(\Rightarrow)$ $A = L L^* = A^*$,
            $(x, L L^* x) = (L^* x, L^* x) > 0$.
        \item $(\Leftarrow)$
            $A = L D L^* = L D^{1/2} D^{1/2} L^* = (L D^{1/2}) (L D^{1/2})^*$.
    \end{itemize}
\end{proof}

Алгоритм вычисления:

\[
    \begin{pmatrix}
        a_{11} & a_{21} & a_{31} \\
        a_{21} & a_{22} & a_{32} \\
        a_{31} & a_{23} & a_{33}
    \end{pmatrix}
    = \begin{pmatrix}
        l_{11} & 0 & 0 \\
        l_{21} & l_{22} & 0 \\
        l_{31} & l_{32} & l_{33}
    \end{pmatrix}
    \begin{pmatrix}
        l_{11} & l_{21} & l_{31} \\
        0 & l_{22} & l_{32} \\
        0 & 0 & l_{33}
    \end{pmatrix}
\]

Решаем матричное уравнение.

Пример разобрали, в целом алгоритм следующий:

\begin{verbatim}
for k = 1, n:
    e[k,k] = sqrt(a[k,k] - e[k,1] ^ 2 - ... - e[k,k-1] ^ 2)
    for i = k + 1, n:
        e[i,k] = (a[i,k] - e[i,1] e[k,1] - ... - e[i,k-1] e[k,k-1]) / e[k,k]
\end{verbatim}

\begin{theorem}
    \[
        |A - \tilde{L} \tilde{L}^*| \le \eps_{\text{machine}} (n + 1)
        \begin{pmatrix} \sqrt{a_{11}} \\ \vdots \\ \sqrt{a_{nn}} \end{pmatrix}
        \begin{pmatrix} \sqrt{a_{11}} & \dots & \sqrt{a_{nn}} \end{pmatrix}
        + O(\eps_{\text{machine}}^2)
    \]
\end{theorem}
