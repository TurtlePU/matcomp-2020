\section{Умножение матриц и вычислительная устойчивость}

\subsection{Сложность матричного умножения}

$O(n^3)$, $O(n^{2,3\ldots})$,..

\subsection{Метод Штрассена}

Позволяет уменьшить кол-во умножений за счёт сложений.

По мастер-теореме сложность $O(n^{\log_2 7})$. Более точно, константа где-то 7,
так что выгоднее обычного умножения только при $n \ge 500$.

\subsubsection{Вывод метода Штрассена}

Вернёмся к блочному умножению.

\[
    \begin{array}{l}
        C_1 = A_1 B_1 + A_2 B_3; \\
        C_2 = A_1 B_2 + A_2 B_4; \\
        C_3 = A_3 B_1 + A_4 B_3; \\
        C_4 = A_3 B_2 + A_4 B_4.
    \end{array}
    \quad\Rightarrow\quad
    C_k = \sum_{i,j = 1}^4 x_{ijk} A_i B_j.
\]

Оказывается, что канонический ранг тензора $x$ равен 7. Получив каноническое
разложение, получаем метод Штрассена.

Точно так же (но увеличивая разбиение) получают асимптотически более эффективные
алгоритмы, но константы там гигантские.

\subsection{Иерархия памяти}

\begin{enumerate}
    \item Регистры;
    \item Кеш;
    \item RAM;
    \item Жесткий диск.
\end{enumerate}

(Отсортировано по убыванию быстродействия, но по возрастанию места.)

Оказывается, что большие матрицы эффективнее всего хранить не по столбцам и не
по строкам, а по блокам, чтобы при выполнении алгоритмов память эффективно
перемещалась по иерархии.

Обратно, стоит формулировать алгоритмы через операции с блоками, чтобы можно
было воспользоваться особенностями иерархии.

(((Обзор LAPACK и вариаций BLAS с оптимизациями под различные архитектуры)))

\subsection{Машинные числа}

(см. презу)

\subsection{Машинные числа: обусловленность}

Пусть $f: X \to Y$.

\[
    f(x + \triangle x) \approx f(x) + f'(x) \triangle x;
\]
\[
    \frac{f(x + \triangle x) - f(x)}{\norm{f(x)}} = \frac{f'(x)}{\norm{f(x)}}
    \cdot \norm{x} \cdot \frac{\triangle x}{\norm{x}}.
\]

Определим число обусловленности (меру обусловленности):

\[
    Cond(f, x) = \frac{\norm{f'(x)}}{\norm{f{x}}} \cdot \norm{x}
\]

\begin{example}
    $f(x) = Ax$, $f'(x) = A$.

    \[
        Cond = \frac{\norm{A}}{\norm{A x}} \norm{x}
        = \frac{\norm{A}}{\norm{y}} \norm{A^{-1} y}
        \le \frac{\norm{A}} \ldots
    \]
\end{example}

Посчитаем обусловленность $f: (A, b) \to A^{-1} b$, то есть $A x = b$.

\begin{lemma}
    Пусть $\norm{\cdot}$ --- матричная норма. Пусть $\norm{A} < 1$. Тогда
    $(I - A)^{-1}$ существует и

    \begin{enumerate}
        \item Ряд Неймана:
            \[
                (I - A)^{-1} = \sum_{k = 0}^\infty A^k
            \]
        \item
            \[
                \norm{(I - A)^{-1}} \le \frac{\norm{I}}{1 - \norm{A}}
            \]
    \end{enumerate}
\end{lemma}

\begin{proof}
    $\dots$
\end{proof}

Теперь посчитаем обусловленность.

\[
    (A + \triangle A) (x + \triangle x) = b + \triangle b;
\]
\[
    \ldots
\]
