\section{Малоранговое приближение матриц}

\subsection{Разделение переменных и скелетное разложение}

\begin{definition}
    Функция с разделенными переменными --- такая функция $f: X \times Y \to Z$,
    что существуют $u$, $v$ такие, что $f(x, y) = u(x) v(y)$.
\end{definition}

Для приближения функций используют сумму функций с разделенными переменными:

\[
    f(x, y) \approx \sum_{i = 1}^r u_i(x) v_i(y)
\]

Например, разложения в ряд Тейлора и в ряд Фурье:

\[
    f(x, y) \approx \sum_{i, j = 0}^p c_{ij} x^i y^j
\]
\[
    f(x, y) \approx \sum_{i, j = 1}^r c_{ij} \sin{\pi i x} \sin{\pi j y}
    \quad \left( (x, y) \in (0, 1)^2 \right)
\]

Как это относится к матричным вычислениям? Возьмём матрицу $A \in \R^{m\times n}$.
$a_{ij}$ --- функция дискретных переменных $i$, $j$.

Если $a_{ij} = u_i v_j$, то $A = u v^T$ (обратное тоже верно).

Раз все строки (столбцы) матрицы коллинеарны, $rk{A} \le 1$.

Примеры: $a_{ij} = \sin{i} \cos{j}$, $a_{ij} = i$.

\begin{definition}
    Скелетное разложение, rank decomposition --- разложение $A = U V^T$ такое,
    что новая размерность $U$, $V$ минимальна.
\end{definition}

Замечания:

\begin{enumerate}
    \item Storage: $m n$ Vs. $(m + n) r$
    \item Разложение единственно с точностью до умножения на обратимую матрицу:
        \[
            U V^T = (U S) (S^{-1} V^T)
        \]
\end{enumerate}

Утверждения:

\begin{enumerate}
    \item $A = U V^T \Rightarrow rk(A) \le r$;
    \item $rk(A) = r \Rightarrow \exists U \in \R^{m \times r},
        V \in \R^{r \times n}: A = U V^T$.
\end{enumerate}

\begin{proof} .

    \begin{enumerate}
        \item очев
        \item очев
    \end{enumerate}
\end{proof}

\subsection{SVD}

\begin{theorem}
    Пусть $A \in \CC^{m \times n}$, $rk(A) = r$, тогда найдутся унитарные
    $U \in \CC^{m \times m}$, $V \in \CC^{n \times n}$ и
    $\sigma_1 \ge \dots \ge \sigma_r > 0$ --- сингулярные числа, что

    \[
        A = U \Sigma V^*,
    \]

    где $\Sigma$ --- диагональная матрица с $\sigma_1, \dots, \sigma_r$ на
    диагонали.
\end{theorem}

\begin{proof}
    Заметим, что $A^* A \ge 0$ и $(A^* A)^* = A^* A$. Из этого следует, что

    \[
        \exists V: V^* A^* A V = diag(\sigma_1^2, \dots, \sigma_n^2),
        \quad \sigma_1 \ge \dots \ge \sigma_n \ge 0. \qquad (1)
    \]

    Рассмотрим $V_r = [v_1, \dots, v_r]$ и
    $\Sigma_r = diag(\sigma_1, \dots, \sigma_r)$, где $\sigma_{r + 1} = 0$
    ($r$ пока неизвестно).

    \[
        V_r^* A^* A V_r = \Sigma_r^2;
    \]
    \[
        \left( \Sigma_r^{-1} V_r^* A^* \right)
        \left( A V_r \Sigma_r^{-1} \right) = I
    \]

    Получается, что $A v_i = \sigma_i u_i$ при $i \in [1, r]$.

    А при $i \in [r + 1, n]$ --- $A v_i = 0$.

    Достраиваем $U_r$ до унитарной $U$:

    \[
        A V = U \begin{pmatrix} \Sigma_r & 0 \\ 0 & 0 \end{pmatrix}
        \quad\Rightarrow\quad
        U^* A V = \begin{pmatrix} \Sigma_r & 0 \\ 0 & 0 \end{pmatrix}.
    \]

    Из этого, в частности, следует, что $r = rk(A)$.
\end{proof}

Замечание:

\begin{enumerate}
    \item $u_i$ --- левые сингулярные векторы;

        $v_i$ --- правые сингулярные векторы;

        $\sigma_{r + 1} = \dots = \sigma_{\min(m, n)} = 0$ --- нулевые
        сингулярные числа.
    \item Сингулярные числа определены однозначно.
    \item Сингулярные векторы определяются однозначно с точностью до множителя
        $C: |C| = 1$ при $\sigma_1 > \dots > \sigma_r > 0$.
    \item
        \[
            Im(A) = \left< u_1, \dots, u_r \right>;
        \]
        \[
            Ker(A) = \left< v_{r + 1}, \dots, v_n \right>.
        \]
    \item SVD $\to$ скелетное разложение:
        \[
            A = U \Sigma V^* = \hat{U} V^*.
        \]

        Переход в другую сторону будет на семинаре.
\end{enumerate}

\begin{theorem} (Эккорта-Янга-Мирского).

    Пусть $k < rk(A)$ и $A_k = U_k \Sigma_k V_k^*$, тогда

    \[
        \min_{rk(B) \le k} \norm{A - B} = \norm{A - A_k}
    \]

    для любой унитарно инвариантной нормы $\norm{\cdot}$, причем
    $\norm{A - A_k}_2 = \sigma_{k + 1}$,
    $\norm{A - A_k}_F = \sqrt{\sum_{i = k + 1}^r \sigma_i^2}$.
\end{theorem}

\begin{definition}
    Нормы Шаттена:

    \[
        \norm{A}_{p, Shatten} = \left( \sum_{i = 1}^r \sigma_i^p \right)^{1/p},
        \quad 1 \le p \le \infty
    \]
\end{definition}

Нормы Шаттена унитарно инвариантны!

Примеры:

\begin{enumerate}
    \item $\norm{\cdot}_{2, Shatten} = \norm{\cdot}_F$;
    \item $\norm{\cdot}_{\infty, Shatten} = \norm{\cdot}_2$;
    \item $\norm{\cdot}_{1, Shatten} = \norm{\cdot}_*$ --- ядерная (nuclear)
        норма.
\end{enumerate}

\subsection{Ортопроектор}

\begin{definition}
    $P$ --- ортопроектор на $L$, если

    \begin{enumerate}
        \item $Im(P) = L$
        \item $P^2 = P$
        \item $P^* = P$
    \end{enumerate}
\end{definition}

\begin{point}
    \[
        \forall x \in \CC^n: P x \perp (I - P) x
    \]
\end{point}

\begin{proof}
    \[
        (P x, (I - P) x) = (x, P^* (I - P) x) = (x, P (I - P) x) = (x, 0) = 0.
    \]
\end{proof}

\begin{point}
    Если $U \in \CC^{n \times k}$, $U^* U = I_k$, то $U U^*$ --- ортопроектор
    на $\left< u_1, \dots, u_k \right>$.
\end{point}

Ортопроекторы, связанные с SVD.

\[
    A = U \Sigma V^*, U = [U_r | \hat{U_r}], V = [V_r | \hat{V_r}]
\]

$U_r U_r^*$ --- ортопроектор на $Im(A)$.

$\hat{V}_r \hat{V}_r^*$ --- ортопроектор на $Ker(A)$.

\subsection{Простейший рандомизированный алгоритм}

Хотим найти $Q \in \R^{m \times r}$ с ортогональными столбцами такую, что для
$A \in \R^{m \times n}$

\[
    A \approx Q Q^T A \quad (\text{если } r = rk(A), Q = U_r, \text{ то
    точное равенство}).
\]

Если мы нашли $Q$, то

\[
    Q (Q^T A) = Q (W \Sigma V^T) = U \Sigma V^T \quad \text{--- SVD.}
\]

Как выбрать $Q$?

\begin{enumerate}
    \item $\Omega = [\omega_1, \dots, \omega_r]$ --- случайная матрица;
    \item $Y = A \cdot \Omega$;
    \item Ортогонализация столбцов $Y$. Например, с помощью Грамма-Шмидта.
        (= QR-разложение)
\end{enumerate}

