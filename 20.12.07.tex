\section{Методы решения частичной задачи на собственные значения}

\[
    A v^{(i)} = \lambda^{(i)} v^{(i)}, \quad A \in \R^{n \times n}, \quad
    \lambda^{(i)} \in \CC
\]

Отношение Релея:

\[
    R(x) = \frac{(A x, x)}{(x, x)}
\]

\[
    \Rightarrow\quad R(v^{(i)})
    = \frac{(A v^{(i)}, v^{(i)})}{(v^{(i)}, v^{(i)})}
    = \lambda^{(i)}
\]

\[
    \nabla R(x) = \frac{2}{(x, x)} (A - R(x)) x \quad\Rightarrow\quad
    \min \lambda^{(i)} = \min_{x \ne 0} R(x); \;
    \max \lambda^{(i)} = \max_{x \ne 0} R(x).
\]

\subsection{Степенной метод (power iteration)}

\[
    x_{k + 1} = \frac{A x_k}{\norm{A x_k}_2}
\]

\subsubsection{Сходимость}

\[
    x_{k + 1} = \frac{A x_k}{\norm{A x_k}_2}
    = \frac{A \frac{A x_{k - 1}}{\norm{A x_{k - 1}}}}
     {\norm{A \frac{A x_{k - 1}}{\norm{A x_{k - 1}}}}}
    = \frac{A^2 x_{k - 1}}{\norm{A^2 x_{k - 1}}} = \dots
    = \frac{A^k x_0}{\norm{A^k x_0}}
\]

Пусть $A$ диагонализуема, и максимальное по модулю собственное значение ---
простое число. Тогда

\[
    x_0 = \sum_{i = 1}^n \alpha_i v^{(i)};
\]

\[
    x_k = \frac{\sum \alpha_i (\lambda^{(i)})^k v^{(i)}}{\norm{\dots}}
    = \left(\frac{\lambda^{(1)}}{|\lambda^{(1)}|}\right)^k
    \frac{v^{(1)} + \sum \frac{\alpha_i}{\alpha_1} \left(\frac{\lambda^{(i)}}
        {\lambda^{(1)}}\right)^k v^{(i)}}{\norm{\dots}}
    = e^{i \phi k} \left( v^{(1)} + \mathcal{O} \left( \left| \lambda^{(2)}
            / \lambda^{(1)} \right|^k \right) \right)
\]

(степень $2k$, если $A = A^\top$)

Замечание. Анализ показывает, почему базисные вектора $\mathcal{K}_k(A, b)$
становятся линейно зависимыми.

\subsubsection{Блочная версия}

\[
    X_0 \in \R^{n \times p}, \quad p \ll n; \qquad
    Y_k = A X_k; \qquad
    Y_k = X_{k + 1} R_{k + 1}
\]

(В последнем выражении выписано QR-разложение)

$L_k = \Im (X_k)$ сходится к инвариантному подпространству $A$. Ошибка
порядка $\mathcal{O}(|\lambda^{(p + 1)} / \lambda^{(p)}|^k)$.

\subsection{Обратная итерация}

\[
    x_{k + 1} = \frac{A^{-1} x_k}{\norm{A^{-1} x_k}_2}
\]

Заметим, что $A^{-1} v^{(i)} = \frac{1}{\lambda^{(i)}} v^{(i)}$. Получаем, что
итерация сходится к собственному вектору наименьшего собственного значения.

\subsubsection{Итерация со сдвигом}

\[
    y_{k + 1} = (A - \sigma I)^{-1} x_k.
\]

Ошибка --- $\mathcal{O} \left( \left| \frac{\lambda^{(1)} - \sigma}
    {\lambda^{(2)} - \sigma} \right|^k \right)$.

\subsubsection{Итерация Релея}

\[
    \sigma = \sigma_k = R(x_k)
\]

Оказывается, что итерация Релея обладает суперсходимостью:
$|e_{k + 1}| = \mathcal{O}(|e_k|^\gamma)$. При $A = A^\top$ гамма равна трём,
иначе двум.

\subsection{Методы Ланцоша и Арнольди}

Вычисляем $A_k x_0$. Почему бы не попытаться найти приближение в пространстве
Крылова $\mathcal{K}_k(A, x_0)$?

\textbf{Алгоритм} (метод Релея-Ритца)

Пусть $Q_k \in \R^{n \times k}$ --- ортогональный базис в $V_k: \dim(V_k) = k$.
Как найти собственные векторы и собственные значения наилучшим образом?

$A_k = Q_k^\top A Q_k$ --- проекционное сужение $A$.

$A_k = S \Lambda S^{-1}$ --- собственное разложение.

$\Lambda = \diag(\theta_1, \dots, \theta_k)$ --- числа Ритца.

$Z_k = Q_k S$ --- векторы Ритца.

\begin{theorem}
    Пусть $Q_k \in \R^{n \times k}: Q_k^\top Q_k = I_k$. Тогда

    \[
        A_k = Q_k^\top A Q_k = \argmin_{R \in \R^{k \times k}}
            \norm{A Q_k - Q_k R}_2
    \]
\end{theorem}

\begin{proof}
    \[
        \norm{A Q_k - Q_k R}_2^2 = \lambda_{\max} \Big((A Q_k - Q_k R)^\top
            (A Q_k - Q_k R)\Big)
    \]

    В свою очередь,

    \begin{multline*}
        (A Q_k - Q_k R)^\top (A Q_k - Q_k R)
        = (A Q_k - Q_k A_k)^\top (A Q_k - Q_k A_k)
        -\\- (A Q_k - Q_k A_k)^\top Q_k Z
        - (Q_k Z)^\top (A Q_k - Q_k A_k) + (Q_k Z)^\top (Q_k Z) =\\=
        (A Q_k - Q_k A_k)^\top (A Q_k - Q_k A_k) + Z^\top Z
    \end{multline*}

    Минимум максимального $\lambda$ достигается при $Z = 0$.
\end{proof}

\textbf{Следствие} Пусть $A = A^\top$ и $A_k = S \Lambda S^\top$. Тогда

\[
    \min_{P_k, D} \norm{A P_k - P_k D}_2
\]

достигается на $D = \Lambda$, $P_k = Q_k S$. Условия на $P_k$, $D$:
$P_k^\top P_k = I$; $\Im(P_k) = \Im(Q_k)$; $D$ --- диагональная.

\subsection{Вектор Фидлера и спектральная бисекция графов}

Дан граф из $n$ вершин. Цель --- ввести $x_k = \{ -1, 1\}$ такие, что

\[
    J(x) = \sum_i \sum_{j \in N(i)} (x_i - x_j)^2 = x^\top L x \to \min,
\]

При следующих условиях:

\[
    \sum x_i = 0; \quad \norm{x}_2^2 = n.
\]
