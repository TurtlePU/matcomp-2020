\section{Методы решения полной задачи на собственные значения}

\subsection{QR-алгоритм}

Не путать с QR-разложением!

\[
    A \in \R^{n \times n}; \quad A v^{(i)} = \lambda^{(i)} v^{(i)}.
\]

\subsubsection{vanilla}

\begin{verbatim}
input A
for k = 1, 2, ...
    Q, R = qr(A)
    A = R @ Q
\end{verbatim}

$A$ сходится к верхне-блочно-трегольной матрице.

Свойства:

\begin{itemize}
    \item
        \[
            A_{k + 1} = R_k Q_k = Q_k^\top A Q_k = \dots =
            (Q_1 \dots Q_k)^\top A (Q_1 \dots Q_k)
        \]
    \item
        \[
            A^k = (Q_1 R_1)^k = Q_1 R_1 \dots Q_1 R_1
            = Q_1 (R_1 Q_1)^{k - 1} R_1 = Q_1 A_2^{k - 1} R_1
            = (Q_1 \dots Q_k) (R_k \dots R_1)
        \]
\end{itemize}

Теперь поймём, что QR-алгоритм --- хорошо замаскированный блочный степенной
метод, т.е.

\begin{verbatim}
for k = 1, 2, ...
    X, _ = qr(A @ X)
\end{verbatim}

В блочном степенном методе $R_{k + 1} X_k^\top = X_{k + 1}^\top A$. Тогда

\begin{multline*}
    A^k = A A^{k - 1} = A X_{k - 1} X_{k - 1}^\top A^{k - 1}
    = X_k R_k X_{k - 1}^\top A^{k - 1}
    = X_k R_k R_{k - 1} X_{k - 2} A^{k - 2} =\\= \ldots
    = X_k (R_k \dots R_1) X_0 A^0 = X_k (R_k \dots R_1) X_0.
\end{multline*}

Таким образом, QR-алгоритм --- блочный степенной метод при $X_0 = I$.

\begin{theorem}
    Пусть

    \begin{enumerate}
        \item $\exists S: A = S \Lambda S^{-1}$, $\det(S[:p,:p]) \ne 0$:
            \[
                \Lambda
                = \begin{pmatrix} \Lambda_p & 0 \\ 0 & \Lambda_q \end{pmatrix}
            \]
        \item
            \[
                |\lambda_1| \ge \dots \ge |\lambda_p| > |\lambda_{p + 1}| \ge
                \dots \ge |\lambda_{p + q}| > 0
            \]
    \end{enumerate}

    Тогда в

    \[
        A_k = \begin{bmatrix}
            A_{11}^{(k)} & A_{12}^{(k)} \\
            A_{21}^{(k)} & A_{22}^{(k)}
        \end{bmatrix}
    \]

    $\forall \eps > 0$:

    \[
        \norm{A_{21}^{(k)}}_2 \le c(\eps) \left|
            \frac{\lambda_{p + 1}}{\lambda_p} + \eps \right|^k
    \]

    Если $\Lambda$ --- диагональная, то $\eps = 0$.
\end{theorem}

\begin{proof}
    oof, надо было поспать подольше.
\end{proof}
