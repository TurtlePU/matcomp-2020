\section{Быстро умножаем векторы на матрицы}

\subsection{Быстрое преобразование Фурье}

(((TODO: пропустил начало)))

\begin{enumerate}
    \item
        \[
            \omega_{2n}^{2pq} = e^{-\frac{2\pi i}{2n} 2pq} = \omega_n^{pq}
        \]

    \item
        \[
            w_{2n}^{2p(n+q)} = w_n^{pq} w_n^{pq}
        \]

    \item ...
\end{enumerate}

\[
    P_{2n} F_{2n} = \begin{pmatrix}
        ...
    \end{pmatrix}
\]

\[
    F_{2 n} X = P_n^{-1} \begin{pmatrix} F_n & 0 \\ 0 & F_n \end{pmatrix}
    \begin{pmatrix} I_n & 0 \\ 0 & W_n \end{pmatrix}
    \begin{pmatrix} I_n & I_n \\ I_n & -I_n \end{pmatrix} X
\]

--- если умножать справа налево по ассоциативности, получится быстро: $O(n)$
на шаг рекурсии, рекурсия от вдвое меньшего $n$. Сложность $O(n \log n)$.

\subsection{Циркулянты}

\begin{definition}
    Матрица называется циркулянтом, если её элементы записываются в следующем
    виде:

    \[
        C = \begin{pmatrix}
            c_0 & c_{n - 1} & c_{n - 2} & \dots & c_1 \\
            c_1 & c_0 & c_{n - 1} & \dots & c_2 \\
            c_2 & . & . & . & . \\
            \vdots & . & . & . & . \\
            c_{n - 1} & c_{n - 2} & . & \dots & c_0
        \end{pmatrix}
    \]
\end{definition}

$y = C x$ --- циклическая свёртка, т.е.

\[
    y_p = \sum_{q = 0}^{n - 1} c_{(p - q) \;\%\; n} \cdot x_q.
\]

Цель: посчитать $C x$ за $o(n^2)$.

Пусть $P$ --- матрица циклического сдвига на 1 (влево? вправо?). Тогда

\[
    C = c_0 P^0 + c_1 P + с_2 P^2 + \ldots + c_{n - 1} P^{n - 1}.
\]

\begin{theorem} Пусть $C \in \CC^{n \times n}$ --- циркулянт, тогда
    \[
        C = F_n^{-1} \diag(F_n c) F_n.
    \]
\end{theorem}

\begin{proof}
    Покажем, что $P_n$ диагонализуется преобразованием Фурье. Тогда теорема
    будет доказана.

    \[
        P_n \begin{pmatrix}
            1 \\ \bar{\omega}_n^q \\ \bar{\omega}_n^{2 q} \\ \vdots \\
            \bar{\omega}_n^{(n - 1) q}
        \end{pmatrix}
        = \begin{pmatrix}
            \bar{\omega}_n^{(n - 1) q} \\ 1 \\ \bar{\omega}_n^q \\ \vdots
        \end{pmatrix}
        = \bar{\omega}_n^{-q} \begin{pmatrix}
            \bar{\omega}_n^{nq} \\ \bar{\omega}_n^q \\ \vdots
        \end{pmatrix}
    \]

    \[
        P_n = F_n^* \Lambda (F_n^*)^{-1},
        \quad \Lambda = \diag(1, \omega_n, \dots, \omega_n^{n - 1}).
    \]

    \[
        \Lambda = \diag(F_n e_1); \Lambda^k = \diag(F_n e_k);
    \]

    \[
        C = c_0 I + \ldots = F_n \diag(F_n C) F_n.
    \]
\end{proof}

Из этого следует

\begin{theorem} дискретная теорема свёртки:
    \[
        C x = F_n^{-1} ( (F_n C) \circ (F_n x)).
    \]
\end{theorem}

\subsection{Двухуровневый циркулянт}

\begin{definition}
    \[
        C = \begin{pmatrix}
            C_0 & C_{n - 1} & \dots & C_1 \\
            C_1 & C_0 & . & \vdots \\
            \vdots & C_1 & . & \vdots \\
            C_{n - 1} & . & . & .
        \end{pmatrix}, C_k \in \CC^{m \times m}
    \]

    ($C_k$ --- циркулянты)
\end{definition}

\begin{theorem}
    Пусть $F_n \otimes F_m = F$, тогда

    \[
        C x = F^{-1} \diag(F C) F.
    \]
\end{theorem}

\begin{proof}
    \[
        C = I_n \otimes C_0 + P \otimes C_1 + P^2 \otimes C_2 + \ldots
    \]

    Далее пользуемся дискретной теоремой свёртки и свойствами Кронекерова
    произведения.
\end{proof}

\subsection{Матрицы Тёплица}

\begin{definition}
    Тёплицева матрица ---

    \[
        T = { t_{i - j} }_{i, j = 1}^n, \quad t_k \in \CC.
    \]
\end{definition}

$y = T x$ или $y_p = \sum_{q = 1}^n t_{p - q} x_q$ --- дискретная свёртка.

Любую Тёплицеву матриу $n \times n$ можно вложить в циркулянт
$(2n - 1) \times (2n - 1)$.

Например, для $n = 3$:

\[
    T = \begin{pmatrix}
        t_0 & t_{-1} & t_{-2} \\
        t_1 & t_0 & t_{-1} \\
        t_2 & t_1 & t_0
    \end{pmatrix};
\]

\[
    C = \begin{pmatrix}
        t_0 & t_{-1} & t_{-2} & | & t_2 & t_1 \\
        t_1 & t_0 & t_{-1} & | & t_{-2} & t_2 \\
        t_2 & t_1 & t_0 & | & t_{-1} & t_{-2} \\
        - & - & - & + & - & - \\
        t_{-2} & t_2 & t_1 & | & t_0 & t_{-1} \\
        t_{-1} & t_{-2} & t_2 & | & t_1 & t_0
    \end{pmatrix}.
\]

Если дополнить $x$ нулями до $x^*$, получим $T x = C x^*$.

Итого,

\verb|Fast matvec: ifft(fft(c) * fft([x, 0].T))|
