\section{Прямые методы решения линейных систем с большими разреженными
матрицами}

\subsection{Формула Шермана-Моррисона}

\[
    A x = b, \qquad A \in \R^{n \times n}, \qquad b \in \R^n.
\]

Пусть $b \to \tilde{b}$. Тогда решение пересчитывается за $O(n^2)$, ведь
LU-разложение уже есть.

Что делать, если изменилась $A$? На случай, если она изменилась каким-то
управляемым способом, и существует формула Шермана-Моррисона.

\begin{point}
    Пусть $\det(A) \ne 0$ и $u, v \in \R^n$. Тогда

    \begin{enumerate}
        \item $A + u v^\top$ обратима $\Leftrightarrow$
            $1 + v^\top A^{-1} u \ne 0$
        \item $(A + u v^\top)^{-1} = A^{-1}
            - \dfrac{A^{-1} u v^\top A^{-1}}{1 + v^\top A^{-1} u}$
    \end{enumerate}
\end{point}

\begin{lemma}
    \[
        \det(I + a b^\top) = 1 + b^\top a, \qquad a, b \in \R^n
    \]
\end{lemma}

\begin{proof}
    \[
        (I + a b^\top) w = w, \quad w \perp b,
        \quad \lambda_1 = \dots = \lambda_{n - 1} = 1.
    \]
    \[
        (I + a b^\top) a = a (1 + b^\top a), \quad \lambda_n = 1 + b^\top a.
    \]
\end{proof}

\begin{proof} (Утверждение 1)
    \begin{enumerate}
        \item \[ \det(A + u v^\top) = \det(A) (1 + v^\top A^{-1} u) \]
        \item Проверим, перемножив $A + u v^\top$ и предполагаемое обратное:
            \begin{multline*}
                (A + u v^\top) \left( A^{-1}
                - \frac{A^{-1} u v^\top A^{-1}}{1 + v^\top A^{-1} u} \right)
                = I - \frac{u v^\top A^{-1}}{1 + \gamma} + u v^\top A^{-1}
                - \frac{u \gamma v^\top A^{-1}}{1 + \gamma}
                =\\= I - \left( \frac{1}{1 + \gamma} - 1
                    + \frac{\gamma}{1 + \gamma} \right) u v^\top A^{-1}
                = I.
            \end{multline*}
    \end{enumerate}
\end{proof}

Подставим формулу из утверждения 1 в решение уравнения
$x = (A + uv^\top)^{-1} b$, чтобы на практике не обращать матрицу:

\[
    x = A^{-1} b - \frac{(A^{-1} u) v^\top (A^{-1} b)}{1 + v^\top (A^{-1} u)}
    = x_1 - \frac{x_2 (v^\top x_1)}{1 + v^\top x_2}.
\]

Замечание: Есть также формула для поправки ранга $r \ge 1$ (формула
Шермана-Вудбери-Моррисона).

\subsection{Форматы представления разреженных матриц}

\subsubsection{COO формат}

3 массива: значения \verb|val|, индексы столбцов \verb|col|, индексы строк
\verb|row|.

Удобно добавлять элементы, но неэффективен для операций, например, matvec
($y = A x$)

\begin{verbatim}
for i in range(nnz(A)):
    y[row[i]] += val[i] * x[col[i]]
\end{verbatim}

\subsubsection{lil (list of lists)}

\verb|col| нет, есть только \verb|row|, который теперь список списков, и в нём
для каждой строки хранятся индексы ненулевых элементов.

\verb|val| тоже список списков, форма совпадает с формой \verb|row|.

\subsubsection{CSR (compressed sparse row)}

\verb|col| и \verb|val| те же, что в COO, но в \verb|row[i]| теперь находится
самый первый индекс элемента из $i$-й строки в массивах \verb|col|, \verb|val|.

Matvec:

\begin{verbatim}
for i in range(n):
    p, q = row[i], row[i + 1]
    y[i] = dot(val[p : q], x[col[p : q]])
\end{verbatim}

\subsection{Заполнение в L и U}

Матрице $A \in \R^{n \times n}$ можно поставить в соответствие ориентированный
граф: $a_{ij} \ne 0$ задаёт наличие ребра из $i$ в $j$.

Для симметричных матриц граф неориентированный.

Модель выше помогает заметить, что правильное сопряжение перестановкой может
сделать $L$ и $U$ в LU-разложении разреженными (при условии разреженности
исходной матрицы).

Для ленточных матриц это вообще неверно.

\subsection{Алгоритм поиска P}

\subsubsection{Алгоритм Катхилла-Макки}

Нужно для $G = (V, E)$ найти такую перестановку номеров
вершин $\sigma$, что метрика $b$ минимальна:

\[
    b = \max_{(x, y) \in E} |\sigma(x) - \sigma(y)|.
\]

Идея состоит в том, чтобы выбрать какую-нибудь вершину первой, а дальше
нумеровать вершины в порядке обхода bfs.

\subsubsection{Minimal degree ordering}
